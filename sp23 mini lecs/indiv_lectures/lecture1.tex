\section{Intro}
\begin{itemize}
    \item My OH is Monday 1-2 and Tuesday 3-4 in Cory 212. 
    \item Email is first.last@
    \item 3rd year cs + math major 
    \item hobbies?
\end{itemize}

\subsection{Some CS70 advice}
\begin{itemize}
    \item Goal: enhance problem solving techniques/approach
    \item Don't fall behind on content, catching up will not be fun
    \item problems, problems, more problems
    \item Ask lots of questions (imperative for strong foundation)
    \item Don't stress, we're in this ride together
\end{itemize}

\section{Propositional Logic}
Relevant notation: \begin{itemize}
    \item $\land$ = and 
    \item $\lor$ = or 
    \item $\neg$ = not 
    \item $\implies$ = implies
    \item $\exists$ = there exists
    \item $\forall$ = forall
    \item $\N$ = natural numbers $\{0, 1, \ldots \}$ 
    \item $a|b$ = $a$ divides $b$
\end{itemize}

$P \implies Q$ is an example of an implication. We can read this as ``If $P$, then $Q$.'' An implication is false only when $P$ is true and $Q$ is false. If $P$ is false, the implication is vacuously true. 

\begin{definition}[Contrapositive]
    If $P \implies Q$ is an implication, then the implication $\neg Q \implies \neg P$ is known as the \textbf{contrapositve}.
\end{definition}

An important identity is that $P \implies Q \equiv \neg Q \implies \neg P$. 

\section{Proofs}
Induction will be in its own section.

Different methods. 

\subsection{Direct proof}
Want to show $P \implies Q$ by assuming $P$ and logically concluding $Q$. 

\subsection{Contraposition}
Want to show $P \implies Q$ by equivalently proving $\neg Q \implies \neg P$. 

\subsection{Contradiction}
Want to show $P$. We do this by assuming $\neg P$ and concluding $R \land \neg R$. 

Why? Idea is that if we can show the implication $\neg P \implies (R \land \neg R)$ is True, this is the same as showing $\neg P \implies F$ is True. The contraposition gives $T \implies P$. 

\subsection{Cases}
Break up a problem into multiple cases i.e. odd vs even. 