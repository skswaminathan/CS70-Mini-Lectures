\section{DIS 10B (Covariance, More Distributions)}

Last time we looked at variance, today we look at 

\subsection{Covariance}
Covariance measures the association between two (or more) RVs $X$ and $Y$. 

Mathematically, \[ \Cov{X,Y} = \E{XY} - \E{X}\E{Y}. \]

We see that if \[ \Cov{X,Y}: \begin{dcases}
     < 0 & \implies \text{X and Y are inversely correlated} \\
     = 0 & \implies \text{no correlation (does not imply independence)} \\
     > 0 & \implies \text{X and Y are directly correlated} \\
\end{dcases} \]

\subsection{Some general principles for Var and Cov}
\begin{itemize}
    \item[1.] $\Cov{X,Y} = \Cov{Y,X}$ 
    \item[2.] $\Cov{X+Y,Z} = \Cov{X,Z} + \Cov{Y,Z}$
    \item[3.] $\Cov{aX, Y} = a\Cov{X,Y}$ 
    \item[4.] $\Var{X} = \Cov{X,X}$ 
    \item[5.] $\Cov{X,Y} = \E{XY} - \E{X}\E{Y}$
    \item[6.] $\Var{X+Y} = \Var{X} + \Var{Y} + 2\Cov{X,Y}$
    
    \vspace{5mm}

    A consequence of above is \textbf{when X and Y are independent}:
    \begin{itemize}
        \item $\Cov{X,Y} = 0$
        \item $\E{XY} = \E{X} \E{Y}$
        \item $\Var{X+Y} = \Var{X} + \Var{Y}$
    \end{itemize}

    \textbf{README: When two variables are independent their covariance is 0 but just because their covariance is 0 doesn't mean they're independent} 
\end{itemize}

\subsection{Poisson Distribution}
We use this distribution when the data tends to fluctuate around some rate or ``average''. We denote this distribution as $\Pois{X}{\lambda}$ where $\lambda$ is the rate. 

We claim \[ \Pr{X = i} = \PoisX{\lambda}{i}. \]

An interesting application is that Binomial distribution as $n \to \infty$ approaches Poisson.

Another interesting fact is that for independent $X$ and $Y$ where $\Pois{X}{\lambda}$ and $\Pois{Y}{\mu}$, then $\Pois{X+Y}{\lambda + \mu}$.  

\subsection{EV/Variance Recap}

\begin{center}
    \begin{tabular}{|c|c|c|}
        \hline 
        $X$ & $\E{X}$ & $\Var{X}$ \\
        \hline
        $\text{Bernoulli}(p)$ & $p$ & $p(1-p)$ \\
        \hline
        $\text{Binomial}(n,p)$ & $np$ & $np(1-p)$ \\
        \hline
        $\text{Geometric}(p)$ & $\frac{1}{p}$ & $\frac{1-p}{p^2}$ \\
        \hline
        $\text{Poisson}(\lambda)$ & $\lambda$ & $\lambda$ \\
        \hline
    \end{tabular}
\end{center}