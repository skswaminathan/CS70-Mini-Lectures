\section{DIS 1A}

\subsection{Induction}
Goal of induction is to show $\forall n P(n)$. 
\subsubsection{(Weak) Induction}
\begin{itemize}
    \item Prove $P(0)$ is true (or relevant base cases), then $\forall n \in \N \left(P(n) \implies P(n+1)\right)$. 
    \item Induction dominoes analogy! 
    \item Sometimes you might have multiple base cases (Problem about $4x+5y$ in Notes 3)
\end{itemize}

\subsubsection{Strengthening the Hypothesis}

Sometimes proving $P(n) \implies P(n+1)$ is not straightforward with induction. In such a scenario, we can try to introduce a (stronger) statement $Q(n)$. We want to construct $Q$ such that $Q(n) \implies P(n)$. Inducting on $Q$ proves $P$. 

\subsubsection{Strong Induction}
\begin{itemize}
    \item Prove $P(0)$ is true (or relevant base cases), then $\forall n \left(\left(P(0) \land P(1) \land \dots \land P(n)\right) \implies P(n+1)\right)$. 
    \item Dominoes analogy, but emphasis on the difference between weak and strong induction (assuming middle domino works vs everything from start to middle).   
\end{itemize}

\subsubsection{Weak vs Strong}

A common point of confusion is when one should use strong induction in lieu of weak induction. Strong induction \textbf{always} works whenever weak induction works. However, there may be scenarios in which the induction hypothesis to prove $n=k+1$ requires more information than just $n=k$. A scenario like this requires strong induction.