\section{DIS 5A}

\subsection{Error Correcting Codes (ECCs)}
Objective: transmit packets of data (integers). 

Two problems may arise. 

\begin{itemize}
    \item[1.] Packets get erased/lost (erasure errors)! If we know we have up to $k$ packet erasures, we fix this by sending $n+k$ packets. 
    \item[2.] Packets get corrupted (general errors)! If we know we have up to $k$ packets corrupted , we fix this by sending $n+2k$ packets. 
\end{itemize}

If we run into erasure errors, we simply use interpolation to recover the lost packets. 

If we run into general errors on the other hand, we need a more powerful tool. 

\subsection{Berlekamp-Welch Algorithm}
We need to identify which indices the error occurs at. Messages are encoded by some polynomial $P(x)$. Our goal is to retrieve $P(x)$. 
\begin{itemize}
    \item[1.] Suppose we know error at $k$ bits. Define the error polynomial \[ E(x) = (x-e_1)(x-e_2)\cdots(x-e_k). \]
    \item[2.] Denote the $i$th packet info we \textbf{see} as $r_i$. Note, $r_i$ may not be the actual value (might be a corrupted value). 
    \item[3.] Solve the equations $P(i)E(i) = r_i E(i)$. 
    \item[4.] Define polynomial $Q(x) := P(x)E(x)$. 
    \item[5.] Substituting, we have \[Q(i) =  P(i) E(i) = r_i E(i). \] 
    \item[6.] We solve linear equations generated by $Q(i) = r_i E(i)$ in step 5 to find the polynomials $E(x), Q(x)$. 
    \item[7.] Once we have that, we can calculate \[ P(x) = Q(x) / E(x). \] 
\end{itemize}
