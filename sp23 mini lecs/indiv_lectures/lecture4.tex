\section{DIS 2A}
\subsection{Graphs}

\subsubsection{Notation}
\begin{itemize}
    \item $V$ denotes set of vertices (points)
    \item $E$ denotes set of edges (lines)
    \item $|V|$ denotes size of set of vertices i.e number of vertices; $|E|$ similarly
    \item Graph $G$ with vertices $V$ and edges $E$ is denoted $G = (V,E)$. 
\end{itemize}

\subsubsection{Vocabulary}

\begin{definition}[Path]
    A \textbf{path} is a sequence of edges. In CS70, we assume a path is \textit{simple} which means no repeated vertices. 
\end{definition}
\begin{definition}[Cycle]
    A \textbf{cycle} is a simple path that starts and ends at the same vertex. 
\end{definition}
\begin{definition}[Walk]
    A \textbf{walk} is any arbitrary connected sequence of edges. 
\end{definition}
\begin{definition}[Tour]
    A \textbf{tour} is a walk that starts and end at the same vertex. 
\end{definition}
\begin{definition}[Connected]
    A graph is \textbf{connected} if there exists a path between any two distinct vertices. 
\end{definition}
\begin{definition}[Eulerian Walk]
    An \textbf{Eulerian walk} is a walk covering all edges without repeating any. 
\end{definition}
\begin{definition}[Eulerian Tour]
    An \textbf{Eulerian tour} is an Eulerian walk that starts and ends at the same vertex. 
\end{definition}

To summarize,

\[\begin{tabular}{rccccc}
    & no repeated vertices & no repeated edges & start = end & all edges & all vertices \\\midrule
Walk & & & & & \\\midrule
Path & \checkmark & \checkmark & & & \\\midrule
Tour & & & \checkmark & & \\\midrule
Cycle & \checkmark$^\ast$ & \checkmark & \checkmark & &\\\midrule
Eulerian Walk & & \checkmark & & \checkmark &\\\midrule
Eulerian Tour & & \checkmark & \checkmark &\checkmark & \\ \midrule 
Hamiltonian Tour & \checkmark & \checkmark & \checkmark & &\checkmark  
\end{tabular} \]

(*except for start and end vertices)

\begin{theorem}[Euler's Theorem]
    An undirected graph $G$ has an Eulerian tour iff $G$ is connected and all its vertices have even degree. 
\end{theorem}

The requires condition for an Eulerian walk is that we have exactly 2 vertices of odd degree. (Of course, the case of 0 odd vertices trivially works since we claim from Euler's Theorem that we can find an Eulerian tour which is a stronger statement than an Eulerian walk)

\begin{definition}[Bipartite]
    A graph is considered bipartite if $V$ can be partitioned into two sets $L$ and $R$ where $V = L \cup R$ such that there are no edges between vertices in $L$ and no edges between vertices in $R$. 
\end{definition}

\subsubsection{The holy grail for graph proofs}

Induct, induct, induct, and induct.

\begin{itemize}
    \item Think about what you want to induct on (edges or vertices???)
    \item Base case (read the problem carefully!)
    \item Prove for $n$ by going from $n \to n-1 \to I.H. \to n$. 
    \begin{itemize}
        \item \textbf{DO NOT} go from $n-1 \to n$ directly.
        \item Why? Build-up error!
        \item Good example of build-up error when trying to prove ``if every vertex of a graph has degree at least 2, then there exists a cycle of length 3.'' Any attempt at induction will give us a false proof but we cannot make square from triangle! 
        \item It's also a logistical nightmare lol (in the times it might accidentally work). Try generating all 5-vertex trees from all 4-vertex trees yikes. 
    \end{itemize}
\end{itemize}

\subsubsection{Relevant Potpourri}

Some other relevant information. 

\begin{definition}[Degree]
    The \textbf{degree} of a vertex $v$ denoted $\deg(v)$ is defined to be the number of incident edges to $v$. 
\end{definition}

\begin{lemma}[Handshake]
    \[ \sum_{v \in V} \deg(v) = 2|E|. \]
\end{lemma}

The idea of a degree (with no adjective) is only well-defined for undirected graphs. We see for directed graphs it's a little funky; we need to introduce the concept of indegree and outdegree. 

In a directed graph, the number of outgoing edges equals the number of ingoing edges. 

We will discuss trees, planarity, coloring, and hypercubes in the next discussion. 