\section{DIS 8A (Conditional Probability)}

So far we've looked at the likelihood of some event $A$ occurring. What if I want to look at the likelihood of some even $A$ occurring given that some event $B$ occurred? 

This is what spurs the insight into conditional probability. The notation ``$A|B$'' should be read as ``$A$ given $B$''. 

\begin{theorem}[Bayes Rule]
    \[ \Pr{A|B} = \frac{\Pr{A \cap B}}{\Pr{B}}. \]
\end{theorem}

A nice corollary of Bayes Rule is \[ \Pr{A|B} = \frac{\Pr{B|A}\Pr{A}}{\Pr{B}}. \]

\subsection{Total Probability}
We explore an idea called the law of total probability. 

\begin{theorem}[Law of Total Probability]
    \begin{align*} \Pr{B} &= \Pr{A \cap B} + \Pr{\overline{A} \cap B} \\
    &= \Pr{B|A}\Pr{A} + \Pr{B|\overline{A}}\Pr{\overline{A}}. 
    \end{align*}
\end{theorem}

\subsection{Independence}
Two events $A$ and $B$ are independent if the occurrence of one does not affect the likelihood of the other. Concretely, for independence TFAE: 
\begin{enumerate}
    \item $\Pr{A \cap B} = \Pr{A}\Pr{B}$
    \item $\Pr{A|B} = \Pr{A}$. 
\end{enumerate}