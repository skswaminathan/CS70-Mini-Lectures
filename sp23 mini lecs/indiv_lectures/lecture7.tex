\section{DIS 3B (More Modular Arithmetic)}

\subsection{Modular inverse}
\begin{lemma}[Bézout]
    For integers $x,y$ such that $\gcd(x,y) = d$, there exist integers $a$ and $b$ that obey \[ ax + by = d. \]
\end{lemma}

We care about the case when $\gcd(x,y) = d = 1$. 

Why? This is how we can find the modular inverse. 

\vspace{3mm}

If $ax + by = 1$, taking$\mod x$ gives us \[ by \equiv 1 \pmod{x} \implies b \equiv y^{-1} \pmod{x}. \] Similarly, taking$\mod y$ gives us \[ ax \equiv 1 \pmod{y} \implies a \equiv x^{-1} \pmod{y}. \]

Takeaway: the values of $a$ and $b$ we will solve for (Q1 on discussion) give us the inverse of $x$ with respect to $y$ and vice versa. 

\subsection{Chinese Remainder Theorem (CRT)}

\begin{theorem}[CRT]
    For pairwise relatively prime integers $m_1, m_2, \ldots, m_n$, the modular system \begin{align*}
        x &\equiv a_1 \pmod{m_1} \\
        x &\equiv a_2 \pmod{m_2} \\
        &\phantom{=}\vdots \\
        x &\equiv a_n \pmod{m_n} 
    \end{align*} has a unique solution $x \pmod{m_1m_2\cdots m_n}$. 

    To clarify, the term pairwise relatively prime means for any distinct $i,j$, it follows $\gcd(m_i, m_j) = 1$. 
\end{theorem}

How do we solve the system above? Discussion Q2...

...or we can solve them a faster way (not taught in the course lol)

\begin{example}
    Suppose we take the first two systems from Q2 on discussion. 

\begin{align*}
        x &\equiv 1\pmod{3} \\
        x &\equiv 3 \pmod{7}. 
    \end{align*}

    Since $\gcd(3,7) = 1$, CRT tells us $x$ has a unique solution mod 21. The first equation tells us there exists some integer $k$ such that $x = 1 + 3k$. Plugging this into the second equation we have \[ 1 + 3k \equiv 3 \pmod{7} \implies k \equiv 3 \pmod{7}. \] Plugging in $k = 3$ gives $x \equiv 10 \pmod{21}$. 

    \vspace{5mm}
    
    If we wanted to solve entirety of Q2 this way, we then apply the same trick above to the systems 
    \begin{align*}
        x &\equiv 10\pmod{21} \\
        x &\equiv 4 \pmod{11}. 
    \end{align*}
\end{example}