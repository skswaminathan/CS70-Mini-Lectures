\section{DIS 13A (Continuous Probability)}

So far we've looked at discrete systems and distributions. Today, we turn our focus more towards continuous models. 

\subsection{Important Definitions}

\begin{definition}[Probability Density Function (PDF)]
    The \textit{probability density function (pdf)} of a random variable $X$ is a function $f: \R \to \R$ where \begin{itemize}
        \item ($\forall x \in \R) f(x) \ge 0$
        \item $\int_{-\infty}^{\infty} f(x) \di x = 1$.
    \end{itemize}
\end{definition}

Visually, $\Pr{a \le X \le b} = \int_{a}^{b} f(x) \di x.$

\textbf{BEWARE:} Tricky nuance about continuous distributions is that $\Pr{X = a} = 0$. This is why we always look at probability of $X$ in a range and not at a singular point. \textbf{THE PDF IS NOT A PROBABILITY}.

\begin{definition}[Cumulative Distribution Function (CDF)]
    The \textit{cumulative distribution function (cdf)} of a random variable $X$ is a function $F$ where \[ F(x) = \Pr{X \le x} = \int_{-\infty}^{x} f(z) \di z. \]
\end{definition}

Remark that as $x \to \infty$, $F(x) \to 1$ and by construction $F(x)$ must be a non-decreasing function since $f(z) \ge 0$. 

(draw the visuals for both cdf and pdf)

\textbf{IMPORTANT RELATION}: $\displaystyle f(x) = \frac{\di}{\di x} F(x)$. In words, the PDF is equal to the derivative of the CDF. 

\subsection{Expectation}
We extend the definition of expectation to \[ \E{X} = \int_{-\infty}^{\infty} x \cdot f_{X}(x) \di x \] where $f_X(x)$ is the pdf of $X$. 

\subsection{Discrete vs Continuous}
\begin{tabular}{|c|c|c|}
    \hline
    Concept & Discrete & Continuous \\
    \hline 
    PDF/PMF & $\Pr{X = x}$ & $f_X(x)$ \\
    \hline 
    CDF & $\Pr{X \le x}$ & $F(x)$ \\ 
    \hline 
    Expectation & $\sum_{x} x \Pr{x=x}$ & $\int_{-\infty}^{\infty} x \cdot f_{X}(x) \di x$ \\
    \hline 
    Variance & $\sum_{x} x^2 \Pr{x=x} - \left(\sum_{x} x \Pr{x=x}\right)^2$ & $\int_{-\infty}^{\infty} x^2 \cdot f_{X}(x) \di x - \left(\int_{-\infty}^{\infty} x \cdot f_{X}(x) \di x\right)^2$\\
    \hline 
    Joint Distribution & $\Pr{X=a, Y=b}$ & $\Pr{a \le X \le b, c \le Y \le d}$ \\
    \hline 
    Independence & $\Pr{X=x, Y=y} = \Pr{X=x}\Pr{Y=y}$ & $f_{X,Y}(x,y) = f_X(x) f_Y(y)$ \\ 
    \hline
\end{tabular}

\subsection{Exponential Distribution}

The exponential distribution is the continuous analog of the geometric distribution. Represents the idea of a ``first arrival'' and also adheres to the memoryless property. Note that $\lambda$ here can be interpreted as the rate of arrival. 

For a random variable $\Expo{X}{\lambda}$, the relevant things we have are \begin{align*} F(x) &= \Pr{X \le x} = 1 - e^{-\lambda x} \\ f_X(x) &= \frac{\di}{\di x} F(x) = \lambda e^{-\lambda x} \\ \E{X} &= \frac{1}{\lambda} \\ \Var{X} &= \frac{1}{\lambda^2}. \end{align*}

\subsection{Big Recap}
\subsubsection{EV/Variance}
\begin{center}
    \begin{tabular}{|c|c|c|}
        \hline 
        $X$ & $\E{X}$ & $\Var{X}$ \\
        \hline
        $\text{Bernoulli}(p)$ & $p$ & $p(1-p)$ \\
        \hline
        $\text{Binomial}(n,p)$ & $np$ & $np(1-p)$ \\
        \hline
        $\text{Geometric}(p)$ & $\frac{1}{p}$ & $\frac{1-p}{p^2}$ \\
        \hline
        $\text{Poisson}(\lambda)$ & $\lambda$ & $\lambda$ \\
        \hline
        $\text{DUniform}(a,b)$ & $\frac{a+b}{2}$ & ? \\
        \hline
        $\text{CUniform}(a,b)$ & $\frac{a+b}{2}$ & $\frac{(b-a)^2}{12}$ \\
        \hline
        $\text{Exponential}(\lambda)$ & $\frac{1}{\lambda}$ & $\frac{1}{\lambda^2}$ \\
        \hline
    \end{tabular}
\end{center}

\subsubsection{Discrete distributions}
\begin{center}
    \begin{tabular}{|c|c|}
        \hline 
        $X$ & $\Pr{X = i}$ \\
        \hline
        $\text{Binomial}(n,p)$ & $\binom{n}{i} p^i (1-p)^n-i$ \\
        \hline
        $\text{Geometric}(p)$ & $p(1-p)^{i-1}$ \\
        \hline
        $\text{Poisson}(\lambda)$ & $\frac{\lambda^i}{i!}e^{-\lambda}$ \\
        \hline
        $\text{DUniform}(a,b)$ & $\frac{1}{b-a}$ \\
        \hline
    \end{tabular}
\end{center}

\subsubsection{Continuous distributions}
\begin{center}
    \begin{tabular}{|c|c|c|}
        \hline
        $X$ & $f_X(X)$ & $F(X)$ \\
        \hline 
        $\text{Uniform}(a,b)$ & $\frac{1}{b-a}$ & $\frac{x}{b-a}$ \\
        \hline
        $\text{Exponential}(\lambda)$ & $\lambda e^{-\lambda x}$ & $1 - e^{-\lambda x}$ \\
        \hline
    \end{tabular}
\end{center}