\section{DIS 3A} 

\begin{definition}[Greatest Common Divisor]
    The \textbf{greatest common divisor} (gcd) of two integers $a, b$ is the greatest $d \in \Z$ such that $d|a$ and $d|b$. 
\end{definition}

How does one efficiently calculate the GCD?

\begin{algothm}[Euclidean Algorithm]
    \begin{algorithmic}
    \Function{GCD}{$a,b$}
    \If{$b = 0$}
        \State \Return $a$
    \EndIf
    \State \Return \Call{GCD}{$b, a \mod b$}
    \EndFunction
    \end{algorithmic}
\end{algothm}

\subsection{Modular Arithmetic}

The relevant notation we'll be using for this section is expressions of the form \[ a \equiv b \pmod{x} \] reads ``$a$ is equivalent to $b$ mod $x$''. It means that the remainder of $a$ when divided by $x$ equals the remainder of $b$ when divided by $x$. 

An important identity is that \[ a \equiv b \pmod{x} \iff (\exists k \in\Z) (a = b + kx). \]

Talk about the ``clock analogy''. 

\begin{example}
    We can see a display of some of the properties:
    \begin{itemize}
        \item Addition: $7 + 4 \equiv 1 \pmod{5}$
        \item Subtraction: $7 - 4 \equiv 1 \pmod{2}$ 
        \item Multiplication: $2 \cdot 3 \equiv 0 \pmod{6}$. 
        \item Division??
    \end{itemize}
\end{example}

In modular arithmetic, division is not well-defined. The opposite of multiplication is multiplying by the modular inverse. 

\begin{definition}[modular inverse]
    The value $a$ is the \textbf{modular inverse} of $x$ with respect to mod $m$ if \[ ax \equiv 1 \pmod{m}. \]
\end{definition}

Does an inverse always exist? No. 

\begin{theorem}
    Let $x$ and $m$ be positive integers. Then $x^{-1} \pmod{m}$ exists and is unique only if $\gcd(x,m) = 1$. 
\end{theorem}