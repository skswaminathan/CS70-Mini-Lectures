\section{DIS 1B}

\subsection{Stable Matching}

Cool application of induction. 

\subsubsection{The Propose and Reject Algorithm}

Suppose jobs proposes to candidates. 

\begin{itemize}
    \item both jobs and candidates have a list of preferences 
    \item every day a job that doesn't have a deal with a candidate will propose to the next best candidate on its preference list
    \item every candidate will tentatively ``waitlist'' the offer from the job (put it on a string)
    \item if a candidate has multiple offers, they will choose the one they prefer the most
    \item the algorithm ends when every candidate has a job on their ``waitlist'' (all these WLs becomes acceptances)
\end{itemize}

(walk through q1 of dis as a class to visualize this)

\subsubsection{Stability}
\begin{definition}[Rogue Couple]
    A job-candidate pair $(J,C)$ is denoted as a \textbf{rogue couple} if they prefer each other over their final assignment in a stable matching instance. 
\end{definition}

\begin{definition}[Unstable]
    A matching that has at least one rogue couple is considered \textbf{unstable}.
\end{definition}

Conversely, a \textbf{stable} matching is one that has no rogue couples. 

Some tricky vocab stuff like stable matching instance. 

\begin{lemma}[Improvement]
    If a candidate has a job offer, then they will always have an offer from a job at least as good as the one they have right now. 
\end{lemma}

Matchings produced by the algorithm are always \textbf{stable}. 

\subsubsection{Optimality}

The propose and reject algorithm is proposer \textit{optimal} and receiver \textit{pessimal}.

\begin{definition}[optimal]
    A pairing is optimal for a group if each entity is paired with who it most prefers  while maintaining stability. 
\end{definition}

Can be thought of a (well that's the best I could do) analogy. 

\begin{definition}[pessimal]
    A pairing is pessimal for a group if each entity is paired with who it least prefers while maintaining stability.
\end{definition}

Can be thought of a (well it can't get worse than this) analogy. 